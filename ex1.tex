\documentclass{article}
\usepackage[utf8]{inputenc}
\usepackage{amsthm}
\usepackage{amsmath}
\usepackage{enumerate}% http://ctan.org/pkg/enumerate


\newtheorem{theorem}{Theorem}[section]
\newtheorem{corollary}[theorem]{Corollary}
\newtheorem{lemma}[section]{Lemma}
\newtheorem{claim}[section]{Claim}
\newtheorem{definition}[theorem]{Definition}
\newtheorem{remark}[theorem]{Remark}
\newtheorem{example}[theorem]{Example}
\newtheorem{assumption}[theorem]{Assumption}
\newtheorem{proposition}[theorem]{Proposition}
\newtheorem{conjecture}[theorem]{Conjecture}
\newtheorem{observation}[theorem]{Observation}


\title{Ex. 1}
\author{Yuval Gitlitz \& Oren Roth}
\date{28.4}

\begin{document}

\maketitle

\begin{enumerate}
	\item
a) Let $G=(V,E),w$ be our graph and the weight function on the edges respectively. We will create bipartite graph $G'=(V\times \{0\},V\times \{1\},E')$ where,
$$E' = \{((u,0),(v,1)): (u,v)\in E \}$$
With weight function $w':E'\rightarrow R$, s.t. $w'(((u,0),(v,1))) = w((u,v))$. We will run weighted prefect matching and receive $M$. We will build the cycle cover accordingly to $M$, cycle by cycle. $c_0$ will be constructed by taking and delete an edge $((u,0),(v,1))$ in $M$ and add $(u,v)$ to $c_0$, go on by take $((v,0),(w,1))$ in $M$, remove it from $M$ and add $(v,w)$ to the cycle until we will reach a node which is matched to $(u,1)$. By then we will finish one cycle and if there are more edges in $M$ we will construct a new cycle $c_1$ and so on until there are no more edges to delete in $M$.
\\ 
b) The algorithm:
\begin{itemize}
	\item Find min cost cycle cover - denoted by $C = (c_1,\ldots,c_k)$. For every $i \in [k]$, define $e_i = (u_i,v_i)$ as an edge in $c_i$.
	\item $G\leftarrow \{(u_k,v_1)\}$
	\item for $i=1$ to $k-1$ do:
	\begin{itemize}
		\item $G \leftarrow G \cup (c_i \setminus \{e_i\} \cup \{u_i,v_{i+1}\})$
	\end{itemize}
	\item $G \leftarrow G \cup (c_i \setminus \{e_i\} \cup \{u_i,v_{i+1}\})$
\end{itemize}
\begin{proof}
We will show :
\begin{enumerate}[I]
	\item $G$ is Hamiltonian cycle. 
	\item cost $G$ is at most $\frac{4}{3}OPT$. 
\end{enumerate}
\begin{enumerate}[I]
	\item We will show the edges in $G$ admit Hamiltonian cycle. We start by $v_1$ and go throug edges of cycle $c_1$ until the node $u_1$ than take the edge $u_1,v_2$ and continue in this fashion until reaching node $u_k$, then taking the edge $\{(u_k,v_1)\}$ and we done,
	\item $cost(C) \le OPT$ because the optimal solution is feasible solution for the cycle cover problem. 
	As each cycle is at least of size of 3 we have that $k\le \frac{|V|}{3}$. $G$ replace $k$ edges of size at least 1 with $k$ edges of size at most $2$, then:
	\begin{align*}
	G \le cost(C) + k \le cost(C) + \frac{|V|}{3} \le OPT +\frac{|V|}{3} \le \frac{4}{3}OPT
	\end{align*}
	And the last inequality is due to the fact that the optimal solution visits $|V|$ edges of weight one at least. 
	
\end{enumerate}	
\end{proof}

\item \begin{enumerate}
	\item We build MST $T = (R,E')$ on the sub graph which includes only nodes in $R$. Our algorithm will return $T$ which is also a feasible solution. We will show $c(T)$ is at most $2OPT$. Let $\tilde T=(\tilde V,\tilde E)$ be the steiner tree which has $c(\tilde T)=OPT$. $c(\tilde T) = \sum_{v\in \tilde V} c(v) + \sum_{e\in \tilde E} c(e)$. In the same way as we showed in class we can have that:
	\begin{align*}
		2 \cdot \sum_{e\in \tilde E} c(e) \ge \sum_{e\in E'} c(e)
	\end{align*}
	 and since $\sum_{v\in R} = 0$ we conclude:
	 $$c(T) = \sum_{v\in R} c(v) + \sum_{e\in E'} c(e) = \sum_{e\in \tilde E} c(e) \le 2OPT$$
	\item Assume towards contradiction that there is exists a $(c\cdot ln |R|)$-approximation algorithm, we will show how to build a reduction based $O(log n)$-approximation algorithm for set cover and we will arrive to contradiction. 
	
\textbf{	The reduction algorithm:}
\begin{enumerate}
	\item
	Given $X =(U,S=\{S_1,\ldots S_m\})$ input for set cover, build the following steiner tree input, $X'=(G=(V,E),R,w)$ where:
	\begin{align*}
	V &= U \cup S\\
	E &= (S\times S) \cup \{(S_i,e_j): e_j\in S_i, S_i\in S\}\\
	R &= U\\
	\forall& e\in E:\quad w(e)=0\\
	\forall& S_i\in S:\quad w(S_i)=w_i\\
	\forall& e_i\in U:\quad w(e_i)=0\\
	\end{align*}
	\item Run the $(c\cdot ln |R|)$-approximation algorithm on $G,R,w$ and receive $T$. 
	\item Return $I = \{S_i : S_i\in V(T); S_i \in S\}$.
\end{enumerate}
We will state two useful lemmas:
		
		\begin{lemma}\label{lemma1}
		Given $\tilde T$ solution to $X'$, $\tilde I = \{S_i : S_i\in V(\tilde T); S_i \in S\}$ is a feasible solution for $X$.
		\end{lemma}
		\begin{proof}
We set $R=U$, and because each node in $R$ is only connected to their sets, hence because is $T$ connected the only way to saturate all the terminal is by taking sets which include all of them - and we conclude $I$ is a valid solution to the set cover problem.
		\end{proof}
\begin{lemma}\label{lemma2}
Given $\tilde T$ solution to $X'$, $\tilde I = \{S_i : S_i\in V(\tilde T); S_i \in S\}$ has the same weight of $\tilde T$ in $X$.
\end{lemma}
\begin{proof}
	The weight of nodes in $S$ is the same as the weight of  the set cover weights, all the other nodes and edges are of weight zero. Therefore:
	$$w(\tilde T) = w(\{S_i : S_i\in V(\tilde T); S_i \in S\}) = w(\tilde I) $$
\end{proof}
	\begin{claim}
		The algorithm is $O(log n)$-approximation algorithm for set cover.
	\end{claim}
	\begin{proof}
		By Lemma \ref{lemma1} $I$ is feasible solution and by Lemma \ref{lemma2} we know $w(I)= w(T)$. Denote by $O_{steiner},O_{set-cover}$ the optimal solutions values of $X',X$ respectively. We conclude:
		\begin{align*}
		w(I) = w(T) \stackrel{(i)}{\leq} (c\cdot ln |R|)\cdot O_{steiner} = (c\cdot ln |R|)\cdot O_{set-cover} 
		\end{align*}
(i) is due to our assumption and the last equality is due to Lemma \ref{lemma2}.
	\end{proof}
As $|R|=n$ we found an $O(log n)$-approximation algorithm for set-cover which accordingly to what we learn in class could happen only if $P=NP$.
	\item
\end{enumerate}
\item 
\item \begin{enumerate}
	\item Let $G=(V,E)$ be a graph we will show the claim holds by induction on $|V|$. Base: $|V| =0$ trivial. Assume that when $|V| < n$ the claim holds. Let be $G$ be a graph with maximum degree $\Delta$, with $|V|=n$ and let $v\in V$. Let $G' = G - v$. $|V(G')| = n - 1$ and its maximum degree at most $\Delta$. We use the induction hypothesis in order to color $G'$ with $\Delta + 1$ colors.
Use the same coloring used for $G'$ in $G$ for all the vertices except $v$. For $v$, it has at most $\Delta$ neighbors and it can be colored using a different color than its neighbors. We used at most $\Delta + 1$ to color the vertices in $G$ hence the claim holds.

The algorithm for finding $(\Delta+1)$-coloring will work in a greedy fashion each time choose an uncolored node and color it with an available color. As the maximum degree is $\Delta$ we know we can do it with $\Delta +1$ colors.

Next we will show that a bipartite graph is two colorable. Let $G = (A, B, E)$ be a bipartite graph. We color all the vertices in $A$ using the first color and all the vertices in $B$ using the second color. All the edges in $A$ are disjoint hence we don't have two vertices which use the first color which are connected. The same applies for $B$ and the second color.
	\item
	\begin{enumerate}
		\item While there exist $v \in V(G)$ such that $\deg(v) \geq \sqrt{n}$
		\begin{enumerate}
			\item Color its neighbors using two colors
			\item $G \leftarrow G - N(v)$
		\end{enumerate}
		\item color $G$ using $\sqrt{(n)} + 1$ colors
	\end{enumerate}
	

	\begin{claim}
		The algorithm run in polynomial time
\end{claim}
\begin{proof}
	First, let us show that step $A$ can be done in polynomial time. The neighborhood of any vertex $v$ in the graph can be two colored because each subgraph is three colored, and if we used one color for $v$, the neighborhood can be two colored. Two colored subgraph is also a two bipartite graph, hence we can use the previous question to color it using 2 colors.
	
	The loop in step $i$ runs at most $\sqrt{n}$ times because each iteration, we remove at least $\sqrt{n}$ vertices from $G$. Additionally, each iteration run polynomial time. Hence, the total run time of step $i$ is polynomial.
	
	Step $ii$ runs in polynomial time using the algorithm from the previous section.
\end{proof}

\begin{claim}
	The algorithm is using $O(\sqrt n)$ colors
\end{claim}
\begin{proof}
	In each iteration of loop $i$, we use two colors. There are at most $\sqrt{n}$ iteration, hence for step $i$ we use $2 \sqrt n = O(\sqrt n)$ colors.
	For step $i$ we used $\sqrt n + 1$ colors.
	For that reason, the total number of colors used by the algorithm is $O(\sqrt n)$.
\end{proof}
\end{enumerate}
\end{enumerate}

\end{document}
